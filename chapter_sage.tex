\chapter{Introduction}\index{Introduction}

\section{What is Sage?}

Algorithms in this Notes are implemented in Python programming language using SageMath (\url{https://www.sagemath.org}).


\begin{dBox}
SageMath is a free open-source mathematics software system licensed under the GPL.
It builds on top of many existing open-source packages: NumPy, SciPy, matplotlib, Sympy, Maxima, GAP, FLINT,
R and many more.
\end{dBox}

You can download binaries at \url{http://www.sagemath.org/download.html} for Mac, and Windows.

Note: Binaries for Windows are avaliable up to version 9.3 (late 2021).
For newer versions you will need to install it in WSL. Follow the instructions at\url{https://doc.sagemath.org/html/en/installation/index.html}.


There is also a cloud version available at https://cocalc.com/

% Using `sage -n jupyter` or running `SageMath 9.x Notebook` application (on Windows) you start jupyter notebook server on localhost:8888.

\noindent Documentation can be found at \url{https://doc.sagemath.org/html/en/index.html}.

\noindent We will moslty use \emph{graph theory} package \url{https://doc.sagemath.org/html/en/reference/graphs/index.html}

\section{Some examples of Sage Graph Theory objects and methods}

For representing undirected graphs we use the \texttt{Graph} class, while for representing directed graphs we use the \texttt{DiGraph} class.

\subsection{Undirected graphs}

Undirected graph is represented using \texttt{Graph} class.

\begin{sageCell}
    G = Graph({0:[1,2,3], 4:[0,2], 6:[1,2,3,4,5]})
\end{sageCell}
There are many methods to access the graph properties. For example, to get a list of vertices use \texttt{vertices} method.
\begin{sageCell}
    G.vertices()
\end{sageCell}
\begin{outCell}
    [0,1,2,3,4,5,6]
\end{outCell}


To display the graph, simply execute a cell with the graph variable name.
\begin{sageCell}
    G
\end{sageCell}

The output is a graphical representation of the graph. If we do not specify vertex coordinates (see below), Sage will use a spring embedder layout algorithm to compute the coordinates.
\begin{outImage}
    \includegraphics[width=6cm]{Images/Introduction/output_96_0.png}
\end{outImage}

If a graph is too large, it will not be displayed. In this case, or if you need to specify other display options, you can use the \texttt{plot} method. There are many options for the \texttt{plot} method, see \url{https://doc.sagemath.org/html/en/reference/plotting/sage/graphs/graph_plot.html} for details.

\medskip
For example. we can specify vertex colors using a dictionary, where keys are colors and values are lists of vertices.
\begin{sageCell}
    G.plot(vertex_colors={'red':[1],'green':[0,4]})
\end{sageCell}
\begin{outImage}
    \includegraphics[width=6cm]{Images/Introduction/output_96_1.png}
\end{outImage}

\subsubsection{Some well-known graphs and graph families}

The famous Petersen graph.
\begin{sageCell}
    graphs.PetersenGraph()
\end{sageCell}
\begin{outImage}
    \includegraphics[width=8cm]{Images/Introduction/output_petersen.png}
\end{outImage}

Complete graphs $K_n$.
\begin{sageCell}
    graphs.CompleteGraph(7)
\end{sageCell}
\begin{outImage}
    \includegraphics[width=8cm]{Images/Introduction/output_complete_7.png}
\end{outImage}

Cycle graphs $C_n$.
\begin{sageCell}
    graphs.CycleGraph(10)
\end{sageCell}
\begin{outImage}
    \includegraphics[width=8cm]{Images/Introduction/output_cycle_10.png}
\end{outImage}

Complete bipartite graphs $K_{n,m}$.
\begin{sageCell}
    graphs.CompleteBipartiteGraph(4, 10)
\end{sageCell}
\begin{outImage}
    \includegraphics[width=0.8\linewidth]{Images/Introduction/output_complete_bipartite_4_10.png}
\end{outImage}

Grid graphs $G_{n,m}$.
\begin{sageCell}
    GG = graphs.GridGraph([4, 5])
    GG.plot()
\end{sageCell}
\begin{outImage}
    \includegraphics[width=0.8\linewidth]{Images/Introduction/output_grid_4_5.png}
\end{outImage}

\subsubsection{Randomly generated graphs}

Random graph on 10 nodes. Each edge is inserted independently with probability 0.3.
\begin{sageCell}
    graphs.RandomGNP(10, 0.3)
\end{sageCell}
\begin{outImage}
    \includegraphics[width=0.8\linewidth]{Images/Introduction/output_random_10_03.png}
\end{outImage}

\subsubsection{Graph constructors}

From a list of edges.
\begin{sageCell}
    Graph([(1,2),(2,3),(3,1),(4,5)])
\end{sageCell}
\begin{outImage}
    \includegraphics[width=10cm]{Images/Introduction/output_edge_list.png}
\end{outImage}

From an adjacency matrix.
\begin{sageCell}
    m = matrix([[int(i != j) for i in range(4)] for j in range(4)])
    m
\end{sageCell}
\begin{outCell}
    [0 1 1 1]
    [1 0 1 1]
    [1 1 0 1]
    [1 1 1 0]
\end{outCell}

\begin{sageCell}
    Graph(m)
\end{sageCell}
\begin{outImage}
    \includegraphics[width=6cm]{Images/Introduction/output_adjacency_matrix.png}
\end{outImage}

Graph to adjacency matrix.
\begin{sageCell}
    M = G.adjacency_matrix()
    m
\end{sageCell}
\begin{outCell}
    [0 1 1 1 1 0 0]
    [1 0 0 0 0 0 1]
    [1 0 0 0 1 0 1]
    [1 0 0 0 0 0 1]
    [1 0 1 0 0 0 1]
    [0 0 0 0 0 0 1]
    [0 1 1 1 1 1 0]
\end{outCell}

From/to graph6 format (compressed string representation of a graph).
\begin{sageCell}
    G = Graph('IheA@GUAo')
    G.plot()
\end{sageCell}
\begin{outImage}
    \includegraphics[width=6cm]{Images/Introduction/output_graph6.png}
\end{outImage}
\begin{sageCell}
    G.graph6_string()
\end{sageCell}
\begin{outCell}
    'IheA@GUAo'
\end{outCell}

Query a graph from local database
\url{http://doc.sagemath.org/html/en/reference/graphs/sage/graphs/graph_database.html}.
For example to get a list of all graphs on 7 vertices with diameter 5.
\begin{sageCell}
    Q = GraphQuery(display_cols=['graph6'], num_vertices=7, diameter=5)
    Q.show()
\end{sageCell}
\begin{outCell}
    Graph6
    --------------------
    F?`po
    F?gqg
    F@?]O
    F@OKg
    F@R@o
    FA_pW
    FEOhW
    FGC{o
    FIAHo
\end{outCell}

\subsection{Basic graph manipulation}

\begin{sageCell}
    G = Graph({0:[1,2,3], 4:[0,2], 6:[1,2,3,4,5]});
\end{sageCell}

\subsubsection*{Access edges, verices, neighbors, etc.}

Access edges.
\begin{sageCell}
    G.edges(labels=False)
\end{sageCell}
\begin{outCell}
    [(0,1),(0,2),(0,3),(0,4),(1,6),(2,4),(2,6),(3,6),(4,6),(5,6)]
\end{outCell}
Note: Edges can have labels. To get a list of edges without labels, use \texttt{labels=False} option. Without this option we get
\begin{outCell}
    [(0,1,None),(0,2,None),(0,3,None),(0,4,None),(1,6,None),(2,4,None),
    (2,6,None),(3,6,None),(4,6,None),(5,6,None)]
\end{outCell}

To check if there is an edge between two vertices use
\begin{sageCell}
    G.has_edge(1,2)
\end{sageCell}
\begin{outCell}
    False
\end{outCell}

Access vertices.
\begin{sageCell}
    G.vertices()
\end{sageCell}
\begin{outCell}
    [0,1,2,3,4,5,6]
\end{outCell}

Access neighbors of a vertex.
\begin{sageCell}
    G.neighbors(0)
\end{sageCell}
\begin{outCell}
    [1,2,3,4]
\end{outCell}

Degree of a vertex is a number of its neighbors
\begin{sageCell}
    G.degree(0)
\end{sageCell}
\begin{outCell}
    4
\end{outCell}

To list degrees of all vertices use
\begin{sageCell}
    G.degree()
\end{sageCell}
\begin{outCell}
    [4,3,3,2,3,2,5]
\end{outCell}

Access number of vertices, edges
\begin{sageCell}
    [G.num_verts(),G.num_edges()]
\end{sageCell}
\begin{outCell}
    [7,10]
\end{outCell}

\subsubsection*{Add/remove vertices, edges}

Add a vertex. Note that the vertices of a graph can be any \emph{hashable} objects, not just integers.
\begin{sageCell}
    G.add_vertex('a')
\end{sageCell}


Method \verb|add_vertex| without arguments adds a single vertex with the smallest available label.
\begin{sageCell}
    newv = G.add_vertex()
    newv
\end{sageCell}
\begin{outCell}
    7
\end{outCell}

\begin{sageCell}
    G.vertices(sort=False)
\end{sageCell}
\begin{outCell}
    ['a',7,0,1,2,3,4,5,6]
\end{outCell}
Note that in certain versions of Sage sorting of vertices by some methods (e.g. \verb|vertices|) is enabled by default and they may fail if the vertices are not comparable. To disable sorting use \verb|sort=False| option.

To add multiple vertices use \verb|add_vertices| method.
\begin{sageCell}
    H=Graph({0:[1,2,3],4:[0,2],6:[1,2,3,4,5]})
    H.add_vertices(range(10,20))
    H.vertices()
\end{sageCell}
\begin{outCell}
    [0,1,2,3,4,5,6,10,11,12,13,14,15,16,17,18,19]
\end{outCell}

To add one edge use \verb|add_edge| method, to add multiple edges use \verb|add_edges| method.
\begin{sageCell}
    H.add_edges([(0, i) for i in range(10, 20)])
\end{sageCell}
Note that edges can have labels. To add an edge with a label you need to pass a a triple $u,v,label$ as an argument to \verb|add_edge| method.
\begin{sageCell}
    H.add_edge(7,12,"EDGE")
\end{sageCell}
To plot a graph with edge labels use \verb|edge_labels=True| option.
\begin{sageCell}
    H.plot(edge_labels=True)
\end{sageCell}
\begin{outImage}
    \includegraphics[width=0.5\linewidth]{Images/Introduction/output_add_edge.png}
\end{outImage}
Note that adding an existing vertex (edge) does not result in an error or a warning.

To delete a vertex or and edge use \verb|delete_vertex| and \verb|delete_edge| methods, respectively.
For example:
\begin{sageCell}
    H.delete_vertex(7)
    H.delete_edge(0,10)
\end{sageCell}
Note that deleting a non-existing vertex results in an error while deleting a non-existing edge does not.

\subsection{Directed graphs}

Directed graph is represented using \verb|DiGraph| class.
\begin{sageCell}
    D = DiGraph({0:[1,2,3],4:[0,2],6:[1,2,3,4,5]})
    D.plot()
\end{sageCell}
\begin{outImage}
    \includegraphics[width=6cm]{Images/Introduction/output_digraph.png}
\end{outImage}
Most of the methods for \verb|Graph| class have their counterparts for \verb|DiGraph| class. For example, to add an edge use \verb|add_edge| method.
\begin{sageCell}
    D.add_edge(5,6)
    D.plot()
\end{sageCell}
\begin{outImage}
    \includegraphics[width=4.5cm]{Images/Introduction/output_digraph2.png}
\end{outImage}

Specific methods for \verb|DiGraph| class include \verb|in_degree| and \verb|out_degree| methods to get in-degree and out-degree of a vertex, respectively.  Similarly, in addition to \verb|neighbors| there are \verb|in_neighbors| and \verb|out_neighbors| methods.
\begin{sageCell}
    [D.in_degree(0),D.out_degree(0),degree(0)]
\end{sageCell}
\begin{outCell}
    [1,3,4]
\end{outCell}
\begin{sageCell}
    [D.in_neighbors(0),D.out_neighbors(0),D.neighbors(0)]
\end{sageCell}
\begin{outCell}
    [[4],[1,2,3],[1,2,3,4]]
\end{outCell}
To check connectivity of a directed graph use \verb|is_strongly_connected| method.
\begin{sageCell}
    [D.is_connected(),D.is_strongly_connected()]
\end{sageCell}
\begin{outCell}
    [True,False]
\end{outCell}
To convert a directed graph to an undirected graph use \verb|to_undirected| (or \verb|to_simple|) method.

\subsubsection*{Even more general graphs}

To allow multiple edges and/or loops use options \verb|multiedges=True| and \verb|loops=True| to the \verb|DiGraph| constructor. For example, consider the following graph.
\begin{sageCell}
    MG = DiGraph({},multiedges=True,loops=True)
    MG.add_vertices([1,2])
    MG.add_edges([(1,2),(1,2),(1, 2),(1,1),(1,1)])
    MG
\end{sageCell}
\begin{outImage}
    \includegraphics[width=7cm]{Images/Introduction/output_multigraph.png}
\end{outImage}

\section{Exercises}

\begin{exercise}
    Write a function \verb|remove_max_vertex(G)| which removes a vertex with the largest degree from undirected graph $G$ (any of them, if there are more than one with the largest degree).
\end{exercise}
\begin{exercise}
    Write a function \verb|plot_bipartite| which plots a bipartite graph in a way that vertices of each bipartition are arranged on two parallel lines.
\end{exercise}
For example:
\begin{sageCell}
    HGR=graphs.HeawoodGraph()
    plot_bipartite(HGR)
\end{sageCell}
\begin{outImage}
    \includegraphics[width=0.4\linewidth]{Images/Introduction/output_bipartite.png}
\end{outImage}

\begin{exercise}
    Write a function \verb`set_random_edge_labels(G,a,b)` which sets edge labels of $G$ to random integers from interval $[a, b]$.

    Write a function \verb`mark_shortest_path(G,a,b)` which calculates a shortest path between the vertices $a$ and $b$ in the  weighted graph $G$ and colors it with red color. (For calculating shortest paths use built-in function \verb`shortest_path`.
\end{exercise}
Example:
\begin{sageCell}
    X=graphs.RandomGNP(20,0.2)
    set_random_edge_labels(X,1,10)
    mark_shortest_path(X,4,7)
\end{sageCell}
\begin{outImage}
    \includegraphics[width=0.4\linewidth]{Images/Introduction/output_shortest_path.png}
\end{outImage}




