\chapter{Maximum independent set}

Write a function \verb`max_independent_set(G, T, B)` which computes a maximum independent set of a graph $G$ with a nice tree decomposition $(T, B)$ (see previous chapter). Algorithm should perform well on graphs with small tree width.

\subsection*{Solution}

\begin{sageCell}
def max_independent_set(G, DT, r, B):
    """Returns maximal independent set of graph G with nice tree decomposition DT, r, B (directed tree with root r and
    bucket map B)"""
    maxI = None
    MEM = {}  # Memorize all results of max_independent_set_rec
    # Dictionary of independent sets for all bags
    rsets = independent_sets(G, B[r])
    # Take maximum over all possible independent sets of the root bag
    for S in rsets:
        IS = max_independent_set_rec(G, DT, B, r, S, MEM)
        if IS != None:
            if maxI == None or len(IS) > len(maxI):
                maxI = IS
    return maxI

def max_independent_set_rec(G, DT, B, t, S, MEM):
    """Returns the largest independent set of the subgraph of G induced on bags below (including) t
    which contains all vertices from S (More precisely, vertices of this independent set from B[t] are exactly vertices S).
    Parameters:
      G        graph
      (DT, B)  tree decomposition with tree directed towards the root
      t        tree vertex
      S        a independent set of bag B[t]"""
    if (t, tuple(S)) in MEM:
        return MEM[(t, tuple(S))]  # If we already calculated the result
    result = None
    sons = DT.neighbors_in(t)
    Bt = B[t]
    if len(sons) == 0: # leaf, len(Bt) == 1
        return S
    elif len(sons) == 1:
        s = sons[0]
        Bs = B[s]
        if len(Bs) > len(Bt): # t is forget node
            u = list(Bs - Bt)[0]  # len(diff) = 1
            result = max_independent_set_rec(G, DT, B, s, S, MEM)
            Su = S | set([u])
            if is_independent_set(G, Su):
                result2 = max_independent_set_rec(G, DT, B, s, Su, MEM)
                if len(result2) > len(result):
                    result = result2
        else: # len(Bs) < len(Bt) - t is introduce node
            u = list(Bt - Bs)[0]  # len(diff) = 1
            Su = S - set([u])
            result = max_independent_set_rec(G, DT, B, s, Su, MEM)
    elif len(sons) == 2:  # join node
        s1 = sons[0]
        s2 = sons[1]
        result1 = max_independent_set_rec(G, DT, B, s1, S, MEM)
        result2 = max_independent_set_rec(G, DT, B, s2, S, MEM)
        result = result1 | result2
    else:
        # should not happen
        return None
    result = result | S
    MEM[(t, tuple(S))] = result  # Memorize the result
    return result
\end{sageCell}
Auxilliary function:
\begin{sageCell}
def independent_sets(G, X):
    """Returns all independent sets of the subgraph of G induced on the set X; exhaustive search, suitable for small X"""
    if len(X)==0:
        return [set([])]
    else:
        X1 = copy(X)
        X2 = copy(X)
        v = X1.pop()
        X2.remove(v)
        Nv = [w for w in G.neighbors(v) if w in X]
        for w in Nv:
            X2.remove(w)
        C1 = independent_sets(G, X1)
        C2 = independent_sets(G, X2)
        C2 = [i.union([v]) for i in C2]
        return C1 + C2
\end{sageCell}

\subsection*{Tests}

\begin{sageCell}
def is_independent_set(G, X):
    return G.subgraph(X).num_edges() == 0
\end{sageCell}

\begin{sageCell}
    G1 = Graph('XTnNw?DOYHgJ@BP@g`wG^PAoa?@C?G??Ga?EG_@oC?NcO?}???P')
    T1, r1, B1 = nice_tree_decomposition(G1)
    MI1 = max_independent_set(G1, T1, r1, B1)
    (is_independent_set(G1, MI1), len(MI1)) == (True, 7)
\end{sageCell}
\begin{outCell}
    True
\end{outCell}

